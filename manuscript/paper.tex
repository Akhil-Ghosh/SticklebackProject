% Options for packages loaded elsewhere
\PassOptionsToPackage{unicode}{hyperref}
\PassOptionsToPackage{hyphens}{url}
\PassOptionsToPackage{dvipsnames,svgnames,x11names}{xcolor}
%
\documentclass[
  12pt,
]{article}
\usepackage{amsmath,amssymb}
\usepackage{lmodern}
\usepackage{iftex}
\ifPDFTeX
  \usepackage[T1]{fontenc}
  \usepackage[utf8]{inputenc}
  \usepackage{textcomp} % provide euro and other symbols
\else % if luatex or xetex
  \usepackage{unicode-math}
  \defaultfontfeatures{Scale=MatchLowercase}
  \defaultfontfeatures[\rmfamily]{Ligatures=TeX,Scale=1}
\fi
% Use upquote if available, for straight quotes in verbatim environments
\IfFileExists{upquote.sty}{\usepackage{upquote}}{}
\IfFileExists{microtype.sty}{% use microtype if available
  \usepackage[]{microtype}
  \UseMicrotypeSet[protrusion]{basicmath} % disable protrusion for tt fonts
}{}
\makeatletter
\@ifundefined{KOMAClassName}{% if non-KOMA class
  \IfFileExists{parskip.sty}{%
    \usepackage{parskip}
  }{% else
    \setlength{\parindent}{0pt}
    \setlength{\parskip}{6pt plus 2pt minus 1pt}}
}{% if KOMA class
  \KOMAoptions{parskip=half}}
\makeatother
\usepackage{xcolor}
\usepackage[margin=1in]{geometry}
\usepackage{graphicx}
\makeatletter
\def\maxwidth{\ifdim\Gin@nat@width>\linewidth\linewidth\else\Gin@nat@width\fi}
\def\maxheight{\ifdim\Gin@nat@height>\textheight\textheight\else\Gin@nat@height\fi}
\makeatother
% Scale images if necessary, so that they will not overflow the page
% margins by default, and it is still possible to overwrite the defaults
% using explicit options in \includegraphics[width, height, ...]{}
\setkeys{Gin}{width=\maxwidth,height=\maxheight,keepaspectratio}
% Set default figure placement to htbp
\makeatletter
\def\fps@figure{htbp}
\makeatother
\setlength{\emergencystretch}{3em} % prevent overfull lines
\providecommand{\tightlist}{%
  \setlength{\itemsep}{0pt}\setlength{\parskip}{0pt}}
\setcounter{secnumdepth}{5}
\newlength{\cslhangindent}
\setlength{\cslhangindent}{1.5em}
\newlength{\csllabelwidth}
\setlength{\csllabelwidth}{3em}
\newlength{\cslentryspacingunit} % times entry-spacing
\setlength{\cslentryspacingunit}{\parskip}
\newenvironment{CSLReferences}[2] % #1 hanging-ident, #2 entry spacing
 {% don't indent paragraphs
  \setlength{\parindent}{0pt}
  % turn on hanging indent if param 1 is 1
  \ifodd #1
  \let\oldpar\par
  \def\par{\hangindent=\cslhangindent\oldpar}
  \fi
  % set entry spacing
  \setlength{\parskip}{#2\cslentryspacingunit}
 }%
 {}
\usepackage{calc}
\newcommand{\CSLBlock}[1]{#1\hfill\break}
\newcommand{\CSLLeftMargin}[1]{\parbox[t]{\csllabelwidth}{#1}}
\newcommand{\CSLRightInline}[1]{\parbox[t]{\linewidth - \csllabelwidth}{#1}\break}
\newcommand{\CSLIndent}[1]{\hspace{\cslhangindent}#1}
\usepackage{setspace} \setstretch{1.15} \usepackage{float} \floatplacement{figure}{t}
\ifLuaTeX
  \usepackage{selnolig}  % disable illegal ligatures
\fi
\IfFileExists{bookmark.sty}{\usepackage{bookmark}}{\usepackage{hyperref}}
\IfFileExists{xurl.sty}{\usepackage{xurl}}{} % add URL line breaks if available
\urlstyle{same} % disable monospaced font for URLs
\hypersetup{
  colorlinks=true,
  linkcolor={cyan},
  filecolor={Maroon},
  citecolor={Blue},
  urlcolor={cyan},
  pdfcreator={LaTeX via pandoc}}

\title{~\Large Stickle!}
\author{\large Matthew Stuart \vspace{-1.1mm}\\
\normalsize Department of Mathematics and Statistics \vspace{-1mm}\\
\normalsize Loyola University Chicago \vspace{-1mm}\\
\normalsize Chicago, IL 60660 \vspace{-1mm}\\
\normalsize \href{mailto:mstuart@luc.edu}{\texttt{mstuart@luc.edu}}
\vspace{-1mm}\\
\strut \\
\large Akhil Ghosh \vspace{-1.1mm}\\
\normalsize Department of Mathematics and Statistics \vspace{-1mm}\\
\normalsize Loyola University Chicago \vspace{-1mm}\\
\normalsize Chicago, IL 60660 \vspace{-1mm}\\
\normalsize \href{mailto:aghosh@luc.edu}{\texttt{aghosh@luc.edu}}
\vspace{-1mm}\\
\strut \\
\large Yoel Stuart \vspace{-1.1mm}\\
\normalsize Department of Biology \vspace{-1mm}\\
\normalsize Loyola University Chicago \vspace{-1mm}\\
\normalsize Chicago, IL 60660 \vspace{-1mm}\\
\normalsize \href{mailto:ystuart@luc.edu}{\texttt{ystuart@luc.edu}}
\vspace{-1mm}\\
\strut \\
\large Gregory J. Matthews \vspace{-1.1mm}\\
\normalsize Center for Data Science and Consulting; Department of
Mathematics and Statistics \vspace{-1mm}\\
\normalsize Loyola University Chicago \vspace{-1mm}\\
\normalsize Chicago, IL 60660 \vspace{-1mm}\\
\normalsize \href{mailto:gmatthews1@luc.edu}{\texttt{gmatthews1@luc.edu}}
\vspace{-1mm}}
\date{}

\begin{document}
\maketitle
\begin{abstract}
Evewryone loves the stickle \vspace{2mm}\\
\emph{Keywords}: Stickle
\end{abstract}

\newpage

\hypertarget{sec:intro}{%
\section{Introduction}\label{sec:intro}}

Sexual Dimorphism is interesting because\ldots\ldots{}

Sexual Dimoprhism lit review: Saitta et al.
(\protect\hyperlink{ref-SaittaEtAl2020}{2020})

Studying sexual dimorphism of phenotypes in modern species is a
straightforward endeavor: measure a phenotype of interest and
statistically test for differences between the sexes. However, examining
sexual dimorphism in fossils is complicated substantially by the issue
that sex may not be able to be directly observed from fossil specimens.
For this study, we have two sources of data: 1) Modern stickleback
specimens with observed sex and 2) fossil specimens with sex unobserved.
We view the sex of the fossils as missing data and use multiple
imputation (Little and Rubin
(\protect\hyperlink{ref-little2002statistical}{2002})) to impute the sex
of the fossils using the modern stickleback fish with observed sex to
model the relationship between sex and observed phenotypes. Once sex is
imputed, an Ornstein--Uhlenbeck (OU) model is fit using a Bayesian
framework to look for sexual dimorphism across a variety of stickleback
phenotypes (e.g.~length, vertebrae count?, etc).

The remainder of this manuscript contains a description of the data in
section \ref{sec:data} and a description of our models in
\ref{sec:models}. Section \ref{sec:results} presents a summary of our
results, and we end with our conclusion and future work in section
\ref{sec:conclusions}.

\hypertarget{sec:data}{%
\section{Data}\label{sec:data}}

The data used here consists of a total of 367 extant specimens with
known sex all collected in the last 30 years. Of these, there are 202
and 165 female and male specimens, respectively.

In addition there are 814 fossil specimens from approximately 10.3
million years ago with unknown sex over 18 time periods spaced about
1000 years apart. Table 1 shows the sample size at each of the 18 time
periods. There are at least 22 specimens at each time period with a high
of 67 specimens in period 7.

\begin{table}[ht]
\centering
\begin{tabular}{rl}
  \hline
time & count \\ 
  \hline
  1 & 43 \\ 
    2 & 41 \\ 
    3 & 51 \\ 
    4 & 41 \\ 
    5 & 46 \\ 
    6 & 48 \\ 
    7 & 67 \\ 
    8 & 55 \\ 
    9 & 42 \\ 
   10 & 33 \\ 
   11 & 37 \\ 
   12 & 22 \\ 
   13 & 41 \\ 
   14 & 43 \\ 
   15 & 46 \\ 
   16 & 47 \\ 
   17 & 56 \\ 
   18 & 55 \\ 
   \hline
\end{tabular}
<!-- \label{tab1} -->
\caption{The number of stickleback fossils at each time point.  Sample size at each time point ranges from a low of 22 to a high of 67.}
\end{table}

What covariates do we have in the data: length, what else,

\hypertarget{sec:models}{%
\section{Models}\label{sec:models}}

\hypertarget{imputation-model}{%
\subsection{Imputation model}\label{imputation-model}}

Let \(W\) be sex and \(X\) are covariates. This data set up here is
\(n_{fossil} + n_{modern}\).\\
\(W = (W_{obs},W_{mis})\) where \(W_{obs}\) and \(W_{mis}\) are the
observed and missing parts of the covariate sex. In this setting,
\(W_{obs}\) perfectly corresponds to the sex of the modern specimens,
and \(W_{mis}\) perfectly corresponds to the fossil data. Missing values
of sex are imputed by drawing from the posterior predictive distribution
\(P(W_{mis}|W_{obs}, X)\). Multiple imputation was implemented here
using MICE (CITE) with predictive mean matching. The data were imputed M
= 100 times.

\hypertarget{ou-model}{%
\subsection{OU model}\label{ou-model}}

After imputing sex, we fit an OU model for different phenotypes of
interest. We only look at \(n_{fossil}\) fossil observations.\\
WLOG we choose one of the columns of \(X\) to be the target ophenotype
and our analysis is repeated for all phenotypes of interest.

Let \(X_{gti}\) be a phenotype of interest for the \(i\)-th observation
\(i = 1\ldots,n_{gt}\), in sex \(g = 1, 2\) and \(t = 1, \ldots, 18\).

\[
X_{gti} = \theta_g + u_{gt} + \epsilon_{gti}
\] \[
u_{gt} = \kappa u_{g(t-1)} + \nu_{gt}
\] \[
u_{g1} \sim \mathcal{N}\left(0,\frac{\tau^2}{1-\kappa^2}\right)
\]

\[
\nu_{gt} \sim \mathcal{N}(0,\tau^2)
\]

\[
\epsilon_{gti} \sim \mathcal{N}(0,\sigma^2)
\]

Priors: \[
\sigma \sim \mathcal{N}(0,100)I_{\{\sigma > 0\}}
\] \[
\tau \sim \mathcal{N}(0,100)I_{\{\tau > 0\}}
\]

\[
\kappa \sim \mathcal{N}(0,25)I_{\{-1 < \kappa < 1\}} \\
\] \[
\theta_g \sim \mathcal{N}(54,625)I_{\{\theta_g > 0\}} \\
\]

All models were built using R Core Team
(\protect\hyperlink{ref-R2022language}{2022})

Cornuault (\protect\hyperlink{ref-Cornault2022}{2022}) Bayesian OU
model.

Bayesian Analysis after multiple imputation Zhou and Reiter
(\protect\hyperlink{ref-ZhouReiter2009}{2010}): They recommend using a
large number of imputations. 5 or 10 is too small. We are using M = 100.

\hypertarget{sec:results}{%
\section{Results}\label{sec:results}}

\hypertarget{sec:conclusions}{%
\section{Future work and conclusions}\label{sec:conclusions}}

\hypertarget{acknowledgements}{%
\section*{Acknowledgements}\label{acknowledgements}}
\addcontentsline{toc}{section}{Acknowledgements}

Stickle!

\hypertarget{supplementary-material}{%
\section*{Supplementary Material}\label{supplementary-material}}
\addcontentsline{toc}{section}{Supplementary Material}

All code for reproducing the analyses in this paper is publicly
available at \url{https://github.com/Akhil-Ghosh/SticklebackProject}

\hypertarget{references}{%
\section*{References}\label{references}}
\addcontentsline{toc}{section}{References}

\hypertarget{refs}{}
\begin{CSLReferences}{1}{0}
\leavevmode\vadjust pre{\hypertarget{ref-Cornault2022}{}}%
Cornuault, Josselin. 2022. {``{Bayesian Analyses of Comparative Data
with the Ornstein--Uhlenbeck Model: Potential Pitfalls}.''}
\emph{Systematic Biology} 71 (6): 1524--40.
\url{https://doi.org/10.1093/sysbio/syac036}.

\leavevmode\vadjust pre{\hypertarget{ref-little2002statistical}{}}%
Little, R. J. A., and D. B. Rubin. 2002. \emph{Statistical Analysis with
Missing Data}. Wiley Series in Probability and Mathematical Statistics.
Probability and Mathematical Statistics. Wiley.
\url{http://books.google.com/books?id=aYPwAAAAMAAJ}.

\leavevmode\vadjust pre{\hypertarget{ref-R2022language}{}}%
R Core Team. 2022. \emph{R: A Language and Environment for Statistical
Computing}. Vienna, Austria: R Foundation for Statistical Computing.
\url{https://www.R-project.org/}.

\leavevmode\vadjust pre{\hypertarget{ref-SaittaEtAl2020}{}}%
Saitta, Evan T, Maximilian T Stockdale, Nicholas R Longrich, Vincent
Bonhomme, Michael J Benton, Innes C Cuthill, and Peter J Makovicky.
2020. {``{An effect size statistical framework for investigating sexual
dimorphism in non-avian dinosaurs and other extinct taxa}.''}
\emph{Biological Journal of the Linnean Society} 131 (2): 231--73.
\url{https://doi.org/10.1093/biolinnean/blaa105}.

\leavevmode\vadjust pre{\hypertarget{ref-ZhouReiter2009}{}}%
Zhou, X., and J. Reiter. 2010. {``A Note on Bayesian Inference After
Multiple Imputation.''} \emph{The American Statistician} 64 (2):
159--63.

\end{CSLReferences}

\end{document}
