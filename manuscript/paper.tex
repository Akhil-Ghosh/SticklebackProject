% Options for packages loaded elsewhere
\PassOptionsToPackage{unicode}{hyperref}
\PassOptionsToPackage{hyphens}{url}
\PassOptionsToPackage{dvipsnames,svgnames,x11names}{xcolor}
%
\documentclass[
  12pt,
]{article}
\usepackage{amsmath,amssymb}
\usepackage{lmodern}
\usepackage{iftex}
\ifPDFTeX
  \usepackage[T1]{fontenc}
  \usepackage[utf8]{inputenc}
  \usepackage{textcomp} % provide euro and other symbols
\else % if luatex or xetex
  \usepackage{unicode-math}
  \defaultfontfeatures{Scale=MatchLowercase}
  \defaultfontfeatures[\rmfamily]{Ligatures=TeX,Scale=1}
\fi
% Use upquote if available, for straight quotes in verbatim environments
\IfFileExists{upquote.sty}{\usepackage{upquote}}{}
\IfFileExists{microtype.sty}{% use microtype if available
  \usepackage[]{microtype}
  \UseMicrotypeSet[protrusion]{basicmath} % disable protrusion for tt fonts
}{}
\makeatletter
\@ifundefined{KOMAClassName}{% if non-KOMA class
  \IfFileExists{parskip.sty}{%
    \usepackage{parskip}
  }{% else
    \setlength{\parindent}{0pt}
    \setlength{\parskip}{6pt plus 2pt minus 1pt}}
}{% if KOMA class
  \KOMAoptions{parskip=half}}
\makeatother
\usepackage{xcolor}
\usepackage[margin=1in]{geometry}
\usepackage{graphicx}
\makeatletter
\def\maxwidth{\ifdim\Gin@nat@width>\linewidth\linewidth\else\Gin@nat@width\fi}
\def\maxheight{\ifdim\Gin@nat@height>\textheight\textheight\else\Gin@nat@height\fi}
\makeatother
% Scale images if necessary, so that they will not overflow the page
% margins by default, and it is still possible to overwrite the defaults
% using explicit options in \includegraphics[width, height, ...]{}
\setkeys{Gin}{width=\maxwidth,height=\maxheight,keepaspectratio}
% Set default figure placement to htbp
\makeatletter
\def\fps@figure{htbp}
\makeatother
\setlength{\emergencystretch}{3em} % prevent overfull lines
\providecommand{\tightlist}{%
  \setlength{\itemsep}{0pt}\setlength{\parskip}{0pt}}
\setcounter{secnumdepth}{5}
\newlength{\cslhangindent}
\setlength{\cslhangindent}{1.5em}
\newlength{\csllabelwidth}
\setlength{\csllabelwidth}{3em}
\newlength{\cslentryspacingunit} % times entry-spacing
\setlength{\cslentryspacingunit}{\parskip}
\newenvironment{CSLReferences}[2] % #1 hanging-ident, #2 entry spacing
 {% don't indent paragraphs
  \setlength{\parindent}{0pt}
  % turn on hanging indent if param 1 is 1
  \ifodd #1
  \let\oldpar\par
  \def\par{\hangindent=\cslhangindent\oldpar}
  \fi
  % set entry spacing
  \setlength{\parskip}{#2\cslentryspacingunit}
 }%
 {}
\usepackage{calc}
\newcommand{\CSLBlock}[1]{#1\hfill\break}
\newcommand{\CSLLeftMargin}[1]{\parbox[t]{\csllabelwidth}{#1}}
\newcommand{\CSLRightInline}[1]{\parbox[t]{\linewidth - \csllabelwidth}{#1}\break}
\newcommand{\CSLIndent}[1]{\hspace{\cslhangindent}#1}
\usepackage{setspace} \setstretch{1.15} \usepackage{float} \floatplacement{figure}{t}
\ifLuaTeX
  \usepackage{selnolig}  % disable illegal ligatures
\fi
\IfFileExists{bookmark.sty}{\usepackage{bookmark}}{\usepackage{hyperref}}
\IfFileExists{xurl.sty}{\usepackage{xurl}}{} % add URL line breaks if available
\urlstyle{same} % disable monospaced font for URLs
\hypersetup{
  colorlinks=true,
  linkcolor={cyan},
  filecolor={Maroon},
  citecolor={Blue},
  urlcolor={cyan},
  pdfcreator={LaTeX via pandoc}}

\title{~\Large Stickle!}
\author{\large Matthew Stuart \vspace{-1.1mm}\\
\normalsize Department of Mathematics and Statistics \vspace{-1.1mm}\\
\normalsize Loyola University Chicago \vspace{-1.1mm}\\
\normalsize Chicago, IL 60660 \vspace{-1.1mm}\\
\normalsize \href{mailto:mstuart1@luc.edu}{\texttt{mstuart1@luc.edu}}
\vspace{-1.1mm}\\
\strut \\
\large Akhil Ghosh \vspace{-1.1mm}\\
\normalsize Department of Mathematics and Statistics \vspace{-1.1mm}\\
\normalsize Loyola University Chicago \vspace{-1.1mm}\\
\normalsize Chicago, IL 60660 \vspace{-1.1mm}\\
\normalsize \href{mailto:aghosh@luc.edu}{\texttt{aghosh@luc.edu}}
\vspace{-1.1mm}\\
\strut \\
\large Yoel E. Stuart \vspace{-1.1mm}\\
\normalsize Department of Biology \vspace{-1.1mm}\\
\normalsize Loyola University Chicago \vspace{-1.1mm}\\
\normalsize Chicago, IL 60660 \vspace{-1.1mm}\\
\normalsize \href{mailto:ystuart@luc.edu}{\texttt{ystuart@luc.edu}}
\vspace{-1.1mm}\\
\strut \\
\large Gregory J. Matthews \vspace{-1.1mm}\\
\normalsize Center for Data Science and Consulting; Department of
Mathematics and Statistics \vspace{-1.1mm}\\
\normalsize Loyola University Chicago \vspace{-1.1mm}\\
\normalsize Chicago, IL 60660 \vspace{-1.1mm}\\
\normalsize \href{mailto:gmatthews1@luc.edu}{\texttt{gmatthews1@luc.edu}}
\vspace{-1.1mm}}
\date{}

\begin{document}
\maketitle
\begin{abstract}
Evewryone loves the stickle \vspace{2mm}\\
\emph{Keywords}: Stickle
\end{abstract}

\newcommand{\iid}{\overset{iid}{\sim}}

\newpage

\hypertarget{sec:intro}{%
\section{Introduction}\label{sec:intro}}

Sexual Dimorphism is interesting because\ldots\ldots{}

Sexual Dimoprhism lit review: Saitta et al.
(\protect\hyperlink{ref-SaittaEtAl2020}{2020})

Studying sexual dimorphism of phenotypes in modern species is a
straightforward endeavor: measure a phenotype of interest and
statistically test for differences between the sexes. However, examining
sexual dimorphism in fossils is complicated substantially by the issue
that sex may not be able to be directly observed from fossil specimens.
For this study, we have two sources of data: 1) Modern stickleback
specimens with observed sex and 2) fossil specimens with sex unobserved.
We view the sex of the fossils as missing data and use multiple
imputation (Little and Rubin
(\protect\hyperlink{ref-little2002statistical}{2002})) to impute the sex
of the fossils using the modern stickleback fish with observed sex to
model the relationship between sex and observed phenotypes. Once sex is
imputed, an Ornstein--Uhlenbeck (OU) model is fit using a Bayesian
framework to look for sexual dimorphism across a variety of stickleback
phenotypes (e.g.~length, vertebrae count?, etc).

The remainder of this manuscript contains a description of the data in
section \ref{sec:data} and a description of our models in
\ref{sec:models}. Section \ref{sec:results} presents a summary of our
results, and we end with our conclusion and future work in section
\ref{sec:conclusions}.

\hypertarget{sec:data}{%
\section{Data}\label{sec:data}}

The data used here consists of a total of 367 extant specimens with
known sex all collected in the last 30 years. Of these, there are 202
and 165 female and male specimens, respectively.

In addition there are 814 fossil specimens from approximately 10.3
million years ago with unknown sex over 18 time periods spaced about
1000 years apart. Table 1 shows the sample size at each of the 18 time
periods. There are at least 22 specimens at each time period with a high
of 67 specimens in period 7.

\begin{table}[ht]
\centering
\begin{tabular}{rl}
  \hline
time & count \\ 
  \hline
  1 & 43 \\ 
    2 & 41 \\ 
    3 & 51 \\ 
    4 & 41 \\ 
    5 & 46 \\ 
    6 & 48 \\ 
    7 & 67 \\ 
    8 & 55 \\ 
    9 & 42 \\ 
   10 & 33 \\ 
   11 & 37 \\ 
   12 & 22 \\ 
   13 & 41 \\ 
   14 & 43 \\ 
   15 & 46 \\ 
   16 & 47 \\ 
   17 & 56 \\ 
   18 & 55 \\ 
   \hline
\end{tabular}
\label{tab1}
\caption{The number of stickleback fossils at each time point.  Sample size at each time point ranges from a low of 22 to a high of 67.}
\end{table}

What covariates do we have in the data: length, what else,

\hypertarget{sec:models}{%
\section{Models}\label{sec:models}}

\hypertarget{imputation-model}{%
\subsection{Imputation model}\label{imputation-model}}

Let \(\boldsymbol{W}\) be an \((n_{extant} + n_{fossil}) \times 1\)
vector of the covariate gender of the stickleback fish and
\(\boldsymbol{X}\) be an \((n_{extant} + n_{fossil}) \times k\) matrix
of the \(k\) phenotypes of interest of the fish. Because
\(\boldsymbol{W}\) contains missing data, we further define
\(\boldsymbol{W} = (\boldsymbol{W}_{obs}^T,\boldsymbol{W}_{mis}^T)^T\)
where \(\boldsymbol{W}_{obs}\) and \(\boldsymbol{W}_{mis}\) are the
observed and missing data of the covariate gender, respectively. In this
setting, \(\boldsymbol{W}_{obs}\) is a \(n_{extant} \times 1\) vector
and \(\boldsymbol{W}_{obs}\) is a \(n_{fossil} \times 1\) vector,
perfectly corresponding to the observable gender of the modern extants
and the unobservable gender of the fossil data, respectively.

We impute the missing gender for the fossil data by sampling from the
posterior predictive distribution
\(P(\boldsymbol{W}_{mis}|\boldsymbol{W}_{obs}, \boldsymbol{X})\) using
the multiple imputation with chained equations (MICE) algorithm (Buuren
and Groothuis-Oudshoorn (\protect\hyperlink{ref-MICE}{2011})) with
predictive mean matching. The imputation algorithm is run to obtain a
total of \(M = 100\) completed datasets.

\hypertarget{ou-model}{%
\subsection{OU model}\label{ou-model}}

After imputing sex, we fit an OU model for different phenotypes of
interest. WLOG we choose one of the columns of \(X\) to be the target
phenotype and our analysis is repeated for all phenotypes of interest.

Let \(W_{ti}\) and \(X_{ti}\) be the gender and phenotype of interest
for the \(i\)-th observation in year \(t\) for \(t = 1, \ldots, 18\) and
\(i = 1\ldots,n_{t}\). We assume that

\begin{align}
X_{ti} \overset{iid}{\sim}\left\{\begin{array}{ll} \mathcal{N}(\theta_f + u_{ft},\sigma^2), & W_{ti} = \text{Female} \\ \mathcal{N}(\theta_m + u_{mt},\sigma^2), & W_{ti} = \text{Male} \end{array}\right.
\label{eq:x}
\end{align}

Look at \ref{eq:x}

For \(g \in \{f,m\}\), \(\theta_g\), is the defined as the overall mean
of the phenotype of interest for each gender, and \(u_{gt}\) are the
gender specific random effects, measuring the difference between the
gender and time-specific mean and the overall gender-specific mean,
which we fit to an autoregressive process of lag 1, which is the
discretization of an OU process. For \(t = 2,\ldots,T\) and
\(g \in \{f,m\}\), the AR1 model for is defined as

\begin{align}
u_{g,t} \overset{iid}{\sim}\mathcal{N}(\kappa_g u_{g,t-1} , \tau^2).
\label{eq:ar1}
\end{align}

\(\kappa_g\) represents the gender-specific correlation between the
previous and current time periods' random effects. In addition,
\(\text{cor}(u_{g,t},u_{g,t+h}) = (\kappa_g)^h\) for
\(h \in \mathbb{N}\).

We also set

\begin{align}
u_{g,1} \overset{iid}{\sim}\mathcal{N}\left(0,\frac{\tau^2}{1-\kappa_g^2}\right).
\label{eq:u1}
\end{align}

This model choice is to make the distribution of the random effects
stationary, i.e.~\(p(\mu_{g,s}) = p(\mu_{g,t})\) for
\(s,t \in \{1,\dots,T\}\).

Priors:

\begin{align}
\sigma & \sim \mathcal{N}(0,100)I_{\{\sigma > 0\}} \nonumber \\
\tau & \sim \mathcal{N}(0,100)I_{\{\tau > 0\}} \nonumber \\
\kappa & \overset{iid}{\sim}\mathcal{N}(0.5,1)I_{\{-1 < \kappa_g < 1\}} \nonumber \\
\theta_g & \overset{iid}{\sim}\mathcal{N}(54,400)I_{\{\theta_g > 0\}}
\label{eq:priors}
\end{align}

All models were built using R Core Team
(\protect\hyperlink{ref-R2022language}{2022})

Cornuault (\protect\hyperlink{ref-Cornault2022}{2022}) Bayesian OU
model.

Bayesian Analysis after multiple imputation Zhou and Reiter
(\protect\hyperlink{ref-ZhouReiter2009}{2010}): They recommend using a
large number of imputations. 5 or 10 is too small. We are using M = 100.

\hypertarget{sec:results}{%
\section{Results}\label{sec:results}}

\hypertarget{sec:conclusions}{%
\section{Future work and conclusions}\label{sec:conclusions}}

\hypertarget{acknowledgements}{%
\section*{Acknowledgements}\label{acknowledgements}}
\addcontentsline{toc}{section}{Acknowledgements}

Stickle!

\hypertarget{supplementary-material}{%
\section*{Supplementary Material}\label{supplementary-material}}
\addcontentsline{toc}{section}{Supplementary Material}

All code for reproducing the analyses in this paper is publicly
available at \url{https://github.com/Akhil-Ghosh/SticklebackProject}

\hypertarget{references}{%
\section*{References}\label{references}}
\addcontentsline{toc}{section}{References}

\hypertarget{refs}{}
\begin{CSLReferences}{1}{0}
\leavevmode\vadjust pre{\hypertarget{ref-MICE}{}}%
Buuren, Stef van, and Karin Groothuis-Oudshoorn. 2011. {``Mice:
Multivariate Imputation by Chained Equations in r.''} \emph{Journal of
Statistical Software} 45 (3): 1--67.
\url{https://doi.org/10.18637/jss.v045.i03}.

\leavevmode\vadjust pre{\hypertarget{ref-Cornault2022}{}}%
Cornuault, Josselin. 2022. {``{Bayesian Analyses of Comparative Data
with the Ornstein--Uhlenbeck Model: Potential Pitfalls}.''}
\emph{Systematic Biology} 71 (6): 1524--40.
\url{https://doi.org/10.1093/sysbio/syac036}.

\leavevmode\vadjust pre{\hypertarget{ref-little2002statistical}{}}%
Little, R. J. A., and D. B. Rubin. 2002. \emph{Statistical Analysis with
Missing Data}. Wiley Series in Probability and Mathematical Statistics.
Probability and Mathematical Statistics. Wiley.
\url{http://books.google.com/books?id=aYPwAAAAMAAJ}.

\leavevmode\vadjust pre{\hypertarget{ref-R2022language}{}}%
R Core Team. 2022. \emph{R: A Language and Environment for Statistical
Computing}. Vienna, Austria: R Foundation for Statistical Computing.
\url{https://www.R-project.org/}.

\leavevmode\vadjust pre{\hypertarget{ref-SaittaEtAl2020}{}}%
Saitta, Evan T, Maximilian T Stockdale, Nicholas R Longrich, Vincent
Bonhomme, Michael J Benton, Innes C Cuthill, and Peter J Makovicky.
2020. {``{An effect size statistical framework for investigating sexual
dimorphism in non-avian dinosaurs and other extinct taxa}.''}
\emph{Biological Journal of the Linnean Society} 131 (2): 231--73.
\url{https://doi.org/10.1093/biolinnean/blaa105}.

\leavevmode\vadjust pre{\hypertarget{ref-ZhouReiter2009}{}}%
Zhou, X., and J. Reiter. 2010. {``A Note on Bayesian Inference After
Multiple Imputation.''} \emph{The American Statistician} 64 (2):
159--63.

\end{CSLReferences}

\end{document}
