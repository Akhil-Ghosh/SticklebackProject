% Options for packages loaded elsewhere
\PassOptionsToPackage{unicode}{hyperref}
\PassOptionsToPackage{hyphens}{url}
\PassOptionsToPackage{dvipsnames,svgnames,x11names}{xcolor}
%
\documentclass[
  12pt,
]{article}
\usepackage{amsmath,amssymb}
\usepackage{lmodern}
\usepackage{iftex}
\ifPDFTeX
  \usepackage[T1]{fontenc}
  \usepackage[utf8]{inputenc}
  \usepackage{textcomp} % provide euro and other symbols
\else % if luatex or xetex
  \usepackage{unicode-math}
  \defaultfontfeatures{Scale=MatchLowercase}
  \defaultfontfeatures[\rmfamily]{Ligatures=TeX,Scale=1}
\fi
% Use upquote if available, for straight quotes in verbatim environments
\IfFileExists{upquote.sty}{\usepackage{upquote}}{}
\IfFileExists{microtype.sty}{% use microtype if available
  \usepackage[]{microtype}
  \UseMicrotypeSet[protrusion]{basicmath} % disable protrusion for tt fonts
}{}
\makeatletter
\@ifundefined{KOMAClassName}{% if non-KOMA class
  \IfFileExists{parskip.sty}{%
    \usepackage{parskip}
  }{% else
    \setlength{\parindent}{0pt}
    \setlength{\parskip}{6pt plus 2pt minus 1pt}}
}{% if KOMA class
  \KOMAoptions{parskip=half}}
\makeatother
\usepackage{xcolor}
\usepackage[margin=1in]{geometry}
\usepackage{longtable,booktabs,array}
\usepackage{calc} % for calculating minipage widths
% Correct order of tables after \paragraph or \subparagraph
\usepackage{etoolbox}
\makeatletter
\patchcmd\longtable{\par}{\if@noskipsec\mbox{}\fi\par}{}{}
\makeatother
% Allow footnotes in longtable head/foot
\IfFileExists{footnotehyper.sty}{\usepackage{footnotehyper}}{\usepackage{footnote}}
\makesavenoteenv{longtable}
\usepackage{graphicx}
\makeatletter
\def\maxwidth{\ifdim\Gin@nat@width>\linewidth\linewidth\else\Gin@nat@width\fi}
\def\maxheight{\ifdim\Gin@nat@height>\textheight\textheight\else\Gin@nat@height\fi}
\makeatother
% Scale images if necessary, so that they will not overflow the page
% margins by default, and it is still possible to overwrite the defaults
% using explicit options in \includegraphics[width, height, ...]{}
\setkeys{Gin}{width=\maxwidth,height=\maxheight,keepaspectratio}
% Set default figure placement to htbp
\makeatletter
\def\fps@figure{htbp}
\makeatother
\setlength{\emergencystretch}{3em} % prevent overfull lines
\providecommand{\tightlist}{%
  \setlength{\itemsep}{0pt}\setlength{\parskip}{0pt}}
\setcounter{secnumdepth}{5}
\newlength{\cslhangindent}
\setlength{\cslhangindent}{1.5em}
\newlength{\csllabelwidth}
\setlength{\csllabelwidth}{3em}
\newlength{\cslentryspacingunit} % times entry-spacing
\setlength{\cslentryspacingunit}{\parskip}
\newenvironment{CSLReferences}[2] % #1 hanging-ident, #2 entry spacing
 {% don't indent paragraphs
  \setlength{\parindent}{0pt}
  % turn on hanging indent if param 1 is 1
  \ifodd #1
  \let\oldpar\par
  \def\par{\hangindent=\cslhangindent\oldpar}
  \fi
  % set entry spacing
  \setlength{\parskip}{#2\cslentryspacingunit}
 }%
 {}
\usepackage{calc}
\newcommand{\CSLBlock}[1]{#1\hfill\break}
\newcommand{\CSLLeftMargin}[1]{\parbox[t]{\csllabelwidth}{#1}}
\newcommand{\CSLRightInline}[1]{\parbox[t]{\linewidth - \csllabelwidth}{#1}\break}
\newcommand{\CSLIndent}[1]{\hspace{\cslhangindent}#1}
\usepackage{setspace} \setstretch{1.15} \usepackage{float} \floatplacement{figure}{t}
\ifLuaTeX
  \usepackage{selnolig}  % disable illegal ligatures
\fi
\IfFileExists{bookmark.sty}{\usepackage{bookmark}}{\usepackage{hyperref}}
\IfFileExists{xurl.sty}{\usepackage{xurl}}{} % add URL line breaks if available
\urlstyle{same} % disable monospaced font for URLs
\hypersetup{
  colorlinks=true,
  linkcolor={cyan},
  filecolor={Maroon},
  citecolor={Blue},
  urlcolor={cyan},
  pdfcreator={LaTeX via pandoc}}

\title{~\Large Increased sexual dimorphism evolves in a fossil
stickleback following ecological release from fish piscivores}
\author{\large Allison Ozark\(^1\), Akhil Ghosh\(^1\), Raheyma
Siddiqui\(^1\), Matthew Stuart\(^1\),\\
\large  Samantha Swank\(^1\), Michael A. Bell\(^2\), Gregory J.
Matthews\(^1\), and Yoel E. Stuart\(^1\) \vspace{-1.1mm}\\
\strut \\
\large \(^1\) Loyola University Chicago, Chicago, IL, USA
\vspace{-1.1mm}\\
\strut \\
\large \(^2\) University of California Museum of Paleontology, Berkeley,
CA, USA \vspace{-1.1mm}\\}
\date{}

\begin{document}
\maketitle
\begin{abstract}
Everyone loves the stickle \vspace{2mm}\\
\emph{Keywords}: Stickle
\end{abstract}

\newcommand{\iid}{\overset{iid}{\sim}}

\newpage

\hypertarget{sec:intro}{%
\section{Introduction}\label{sec:intro}}

Ecological release theory suggests that a prey population's niche should
expand when predation is relaxed or removed (reviewed in Herrmann,
Stroud, and Losos (\protect\hyperlink{ref-Herrmann2021}{2021})). Niche
changes in habitat and resource use should then drive phenotypic
evolution as the population adapts to new conditions (Herrmann, Stroud,
and Losos (\protect\hyperlink{ref-Herrmann2021}{2021})). For example,
resource use expansion should result in trophic trait evolution (cite;
e.g.,). Habitat expansion should result in locomotor change (cite;
e.g.,). And release from enemies should result in reduction of costly
defense traits (cite; e.g.,).

Niche expansion could manifest through a combination of increasing
within- and between-individual niche widths (Bolnick et al.
(\protect\hyperlink{ref-Bolnick2010}{2010}); Herrmann, Stroud, and Losos
(\protect\hyperlink{ref-Herrmann2021}{2021})), including divergence
between the sexes (Bolnick and Doebeli
(\protect\hyperlink{ref-Bolnick2003}{2003}); Cooper, Gilman, and
Boughman (\protect\hyperlink{ref-Cooper2011}{2011})). Theory and
empirical data suggest that a population presented with novel ecological
opportunity might experience disruptive selection on males and females
stemming from intraspecific competition over newly accessible resources,
resulting in intersexual divergence in habitat use and associated
phenotypes (Schoener 1968; Shine
(\protect\hyperlink{ref-Shine1989}{1989}); Bolnick and Doebeli
(\protect\hyperlink{ref-Bolnick2003}{2003}); Butler, Sawyer, and Losos
(\protect\hyperlink{ref-Butler2007}{2007}); Bolnick and Lau
(\protect\hyperlink{ref-Bolnick2008}{2008}); Cooper, Gilman, and
Boughman (\protect\hyperlink{ref-Cooper2011}{2011}); but see Stuart et
al. (\protect\hyperlink{ref-Stuart2021}{2021}); Blain
(\protect\hyperlink{ref-Blain2022}{2022})). Such character displacement
between the sexes is therefore one explanation for sexual dimorphism (de
Lisle papers).

We tested this link between release from predators and the evolution of
sexual dimorphism in a well-preserved, high resolution lineage of the
fossil stickleback fish (Gasterosteus doryssus). The sex of the
individual fossils, which is unobserved in practice, was imputed using
multiple imputation by chained equations (MICE) (Buuren and
Groothuis-Oudshoorn (\protect\hyperlink{ref-MICE}{2011})) with observed
data used to train the imputation model based on extant stickleback with
known sex. Then we tracked the multivariate evolution of sexual
dimorphism over \textasciitilde16,000 years for traits related to
swimming, feeding and defense. The fossil sequence captures a bout of
adaptive evolution following ecological release from predators, as
follows. The fossil lineage appeared in the depositional environment as
a fully armored form, with complete pelvic girdles, two pelvic spines,
and three dorsal spines, on average (Michael A. Bell, Travis, and Blouw
(\protect\hyperlink{ref-Bell2006}{2006}); Stuart, Travis, and Bell
(\protect\hyperlink{ref-Stuart2020}{2020})). However, this habitat
appears to have been missing fish piscivores (e.g., trout and other
salmonids known to prey on modern stickleback; M. A. Bell
(\protect\hyperlink{ref-Bell2009}{2009})), relaxing putative selection
for armor (Reimchen papers, Bell papers, Bowne
(\protect\hyperlink{ref-Bowne1994}{1994}); Roesti et al.~2023). The
lineage thus adaptively evolved armor loss and reduction in several
other traits as well (Hunt, Bell, and Travis
(\protect\hyperlink{ref-Hunt2008}{2008}); Stuart, Travis, and Bell
(\protect\hyperlink{ref-Stuart2020}{2020}); Siddiqui et al.~in prep),
and tooth wear data suggest that the population also began eating more
planktonic prey, expanding to an open water lifestyle (Purnell et
al.~2007).

Sexual Dimorphism is interesting because\ldots\ldots{}

Sexual Dimoprhism lit review: Saitta et al.
(\protect\hyperlink{ref-SaittaEtAl2020}{2020})

The remainder of this manuscript contains a description of the data in
section \ref{sec:data} and a description of our models in
\ref{sec:models}. Section \ref{sec:results} presents a summary of our
results, and we end with our conclusion and future work in section
\ref{sec:conclusions}.

\hypertarget{methods-and-materials}{%
\section{Methods and Materials}\label{methods-and-materials}}

A major challenge in this work is inferring the sex of fossil specimens.
This typically cannot be done directly, except for lineages whose sexes
are distinguished by the presence or absence of sex-specific characters
(Hone and Mallon (\protect\hyperlink{ref-Hone2017}{2017}); Saitta et al.
(\protect\hyperlink{ref-SaittaEtAl2020}{2020})). Instead,
paleobiologists resort to statistical detection of sex and sexual
dimorphism, including tests for bimodality in trait distributions (e.g.,
Hone and Mallon (\protect\hyperlink{ref-Hone2017}{2017})) and divergence
in growth curves (e.g., Saitta et al.
(\protect\hyperlink{ref-SaittaEtAl2020}{2020})).

We used a third approach to infer fossil sex and study the evolution of
sexual dimorphism by taking advantage of the fact that G. doryssus is
part of the extant threespine stickleback species complex (Gastersteus
aculeatus). Along with the fossil data collected for this study, we also
have data from extant populations of the Threespine Stickleback,
Gasterosteus aculeatus, measuring the same traits for individuals whose
sex could be determined directly. We then combined these sets of data
and treat the sex of the fossil specimens as missing and used multiple
imputation (Little and Rubin
(\protect\hyperlink{ref-little2002statistical}{2002})) to impute the sex
of the fossils many times. Imputations were performed using the MICE
(Buuren and Groothuis-Oudshoorn (\protect\hyperlink{ref-MICE}{2011}))
algorithm and implemented in R (R Core Team
(\protect\hyperlink{ref-R2022language}{2022})) with M = 100 imputations
performed.

After imputing the sex of the fossil data, each completed data set is
then used to fit a modified Ornstein--Uhlenbeck (OU) (Uhlenbeck and
Ornstein (\protect\hyperlink{ref-OUProcess}{1930})) model using a
Bayesian framework to look for sexual dimorphism across a variety of
stickleback phenotypes

\hypertarget{sec:data}{%
\section{Data}\label{sec:data}}

\hypertarget{fossil-specimen-data}{%
\subsection{Fossil Specimen Data}\label{fossil-specimen-data}}

We used Gasterosteus doryssus data that were previously reported by
Stuart, Travis, and Bell (\protect\hyperlink{ref-Stuart2020}{2020}),
Voje, Bell, and Stuart (\protect\hyperlink{ref-Voje2022}{2022}), and
Siddiqui et al.~(in review). Briefly, the data were collected from
fossil Series K from Quarry D (Cerasoni et al., in review), dug from an
open pit diatomite mine at 9.526° N, 119.094° W, near Fernley, Nevada,
USA. Series K consisted of 18 samples taken at \textasciitilde1000-year
intervals, and mean sample times span \textasciitilde16,363 years. Fish
from series K were measured for 16 ecomorphological traits related to
armor, swimming, and feeding (Table 1). Series K started at a previously
documented horizon when a low armored lineage of stickleback with zero
to one dorsal spines, zero pelvic spines, and highly reduced pelvises
was completely replaced by a high armored lineage of stickleback with
three dorsal spines, two pelvic spines, and complete (Bell et al.~2006;
Bell 2009; Stuart et al.~2020). This lineage subsequently evolved
reduction in armor, body size, and traits related to swimming and
feeding (Bell et al.~2006; Stuart et al.~2020; Siddiqui et al.~in
review). The tempo and mode of armor reduction during this sequence
suggests adaptive evolution by natural selection (Hunt et al.~2008), and
we focus on the multivariate evolution of sexual dimorphism by this
second lineage.

The fossil data consists of 814 specimens with unknown sex over 18 time
periods spaced \textasciitilde1000 years apart. Table 2 shows the sample
size at each of the 18 time periods. There are at least 22 specimens at
each time period with a high of 67 specimens in period 7.

\begin{longtable}[]{@{}
  >{\raggedright\arraybackslash}p{(\columnwidth - 4\tabcolsep) * \real{0.3492}}
  >{\raggedright\arraybackslash}p{(\columnwidth - 4\tabcolsep) * \real{0.1905}}
  >{\raggedright\arraybackslash}p{(\columnwidth - 4\tabcolsep) * \real{0.4603}}@{}}
\caption{Traits and trait descriptions. `sc' denotes size correction of
trait against standard length. Names of bones follow Bowne
(\protect\hyperlink{ref-Bowne1994}{1994}) unless otherwise
noted.}\tabularnewline
\toprule()
\begin{minipage}[b]{\linewidth}\raggedright
Trait Name
\end{minipage} & \begin{minipage}[b]{\linewidth}\raggedright
Trait Code
\end{minipage} & \begin{minipage}[b]{\linewidth}\raggedright
Trait Description
\end{minipage} \\
\midrule()
\endfirsthead
\toprule()
\begin{minipage}[b]{\linewidth}\raggedright
Trait Name
\end{minipage} & \begin{minipage}[b]{\linewidth}\raggedright
Trait Code
\end{minipage} & \begin{minipage}[b]{\linewidth}\raggedright
Trait Description
\end{minipage} \\
\midrule()
\endhead
Standard Length & sl & Distance from anterior tip of premaxilla to
posterior end of last vertebra (hypural plate) \\
Dorsal Spine & mds & Number of dorsal spines from 0 to 3 \\
Dorsal Fin Ray & mdf & Number of bones in the dorsal fin posterior to
the third dorsal spine (i.e., soft dorsal fin rays) \\
Anal Fin Ray & maf & Number of bones in the anal fin posterior to the
anal spine (i.e., soft dorsal fin rays) \\
Abdominal Vertebra & mav & Number of vertebrae anterior to the first
vertebra contacting an anal fin pterygiophore (Aguirre et al.~2014) \\
Caudal Vertebra & mcv & Number of vertebrae posterior to and including
the first vertebra contacting an anal fin pterygiophore (Aguirre et
al.~2014) \\
Pterygiophore number & mpt & Number of pterygiophores anterior to but
excluding the pterygiophore under the third dorsal spine, which is
immediately anterior to and contiguous with the dorsal fin \\
Pelvic Spine length & lps.sc & Length from the base of one pelvic spine
above its articulation with the pelvic girdle to its distal tip \\
Ectocoracoid & ect.sc & Length between the anterior and posterior tips
of the shoulder girdle base (i.e., ectocoracoid) \\
Pelvic Girdle & tpg.sc & Length between the anterior to posterior tips
along midline. If vestigial, the sum of longest anterior-posterior axis
for the vestiges \\
Cleithrum length & cle.sc & Length from free dorsal tip to ventral tip
of the cleithrum on the anterior margin of the shoulder girdle (i.e.,
cleithrum) \\
Premaxilla & pmx.sc & Length from the anterior tip of the premaxilla to
the distal tip of the ascending process of the premaxilla \\
Dorsal Spine & Ds\#.sc\# = 1,2,or 3 & Length from the base of a dorsal
spine above the pterygiophore to its distal tip along the anterior
edge \\
Pterygiophore & lpt.sc & Distance between the anterior to posterior tips
of the pterygiophore immediately preceding the 3rd dorsal spine (when
present) \\
\bottomrule()
\end{longtable}

\begin{table}[ht]
\centering
\caption{The number of stickleback fossils at each time point.  Sample size at each time point ranges from a low of 22 to a high of 67.}
\begin{tabular}{rl}
  \hline
time & count \\ 
  \hline
  1 & 43 \\ 
    2 & 41 \\ 
    3 & 51 \\ 
    4 & 41 \\ 
    5 & 46 \\ 
    6 & 48 \\ 
    7 & 67 \\ 
    8 & 55 \\ 
    9 & 42 \\ 
   10 & 33 \\ 
   11 & 37 \\ 
   12 & 22 \\ 
   13 & 41 \\ 
   14 & 43 \\ 
   15 & 46 \\ 
   16 & 47 \\ 
   17 & 56 \\ 
   18 & 55 \\ 
   \hline
\end{tabular}
\label{tabn}

\end{table}

\hypertarget{extant-specimen-data}{%
\subsection{Extant Specimen data}\label{extant-specimen-data}}

To span the gamut of stickleback diversity for our predictive model, we
sampled modern stickleback from lakes containing generalist stickleback
populations (Hendry et al. (\protect\hyperlink{ref-Hendry2009}{2009});
Bolnick (\protect\hyperlink{ref-Bolnick2011}{2011})) and from lakes
containing benthic-limnetic species pairs (Baumgartner et al.~1988;
Schluter and McPhail (\protect\hyperlink{ref-Schluter1992}{1992}))
(Table 2 (What is this?)). The generalist populations were collected by
YES in 2013 and previously described in Stuart et al.
(\protect\hyperlink{ref-Stuart2017}{2017}). These samples were fixed in
formalin, then stained for bone with Alizarin Red in 2013. Benthic and
limnetic specimens were kindly loaned by D. Schluter and his lab at
University of British Columbia. They collected benthic and limnetic
individuals from Enos Lake in 1988 and from Emily Lake, Little Quarry
Lake, Paxton Lake, and Priest Lake in 2018. The Enos specimens had been
fixed whole in formalin and stored in 40\% isopropanol. The specimens
from the other lakes were initially preserved whole in 95\% ethanol in
the field before being gradually transferred to water then formalin in
the lab and ultimately stored in 40\% isopropanol. In 2019, we stained
these specimens for bone using Alizarin Red.

We next replicated fossil data collection (Table 1) on these extant
specimens. Standard length as well as pelvic-spine length on each side
were measured with calipers. We used a dissection microscope to count
dorsal spines, pelvic spines, dorsal-fin rays, and anal-fin rays. Right
and left-side pelvic girdle lengths and ectocoracoid lengths were
measured from ventral photographs taken using a Canon EOS Rebel T7 with
a Tamron 16-300 mm MACRO lens mounted on a leveled Kaiser RS1 copy
stand. Specimens were held in place for ventral photographs using a
small tabletop vise with an attached scale bar. Lateral X-rays were used
to measure dorsal spine length, number of pterygiophores anterior to the
pterygiophore holding the third spine, length of the pterygiophore just
anterior to the third spine, cleithrum length, and pre-maxilla ascending
branch length. We also counted vertebrae from the X-rays: abdominal
vertebrae were counted anterior to the first vertebra with a haemal
spine contacting an anal fin pterygiophore. Caudal vertebrae were
posterior, including the first vertebra with the haemal spine contacting
the anal fin pterygiophore (following Aguirre, Walker, and Gideon
(\protect\hyperlink{ref-Aguirre2014}{2014}). X-rays were taken with an
AXR Hot Shot X-ray Machine (Associated X-ray Corporation) at the Field
Museum of Natural History. Specimens were exposed at 35kV and 4mA. Small
fish were exposed for 7s, medium fish for 8s, and large fish for 10s. We
developed the film and scanned individual images of each fish using the
B\&W Negatives setting on an Epson Perfection 4990 Photo flatbed at 2400
dpi. Measurements from photographs and X-rays were taken with FIJI
(Schindelin et al.~2012) and its plugin ObjectJ
(\url{https://sils.fnwi.uva.nl/bcb/objectj/}). All photographs, X-rays,
and ObjectJ files have been uploaded to Morphosource.org (accession \#
TBD). We dissected individuals from the generalist populations to
determine sex from the gonads. Individuals from the species-pair lakes
were sexed by Schluter and his lab, using a genotyping protocol (confirm
and cite).

The extant data used here consists of a total of 367 specimens all with
known sex. Of these, there are 202 and 165 female and male specimens,
respectively.

\hypertarget{outlier-analysis.}{%
\subsection{Outlier analysis.}\label{outlier-analysis.}}

To check for outliers, we calculated within-group means and standard
deviations for each trait separately for K series fossil specimens
(pooled across samples) and for extant specimens (pooled across lakes).
We noted trait values greater than 3.5 standard deviations from the mean
as potential outliers. We checked whether these potential outliers were
a result of data entry and collection error and corrected them if they
were. We turned the remaining outlier trait values to NAs. (confirm)
(Wait, what? Why? Anything that was 3.5 SD above the mean was recorded
as missing? How do you justify this? )

Missing data imputation, including fossil sex

Quantification and evolution of sexual dimorphism

What covariates do we have in the data: length, what else,

\hypertarget{sec:models}{%
\section{Models}\label{sec:models}}

\hypertarget{imputation-model}{%
\subsection{Imputation Model}\label{imputation-model}}

Let \(\boldsymbol{W}\) be an \((n_{extant} + n_{fossil}) \times 1\)
vector of the covariate gender of the stickleback fish,
\(\boldsymbol{X}\) be an \((n_{extant} + n_{fossil}) \times K\) matrix
of the \(K\) continuous phenotypes of interest, and \(\boldsymbol{Y}\)
be an \((n_{extant} + n_{fossil}) \times L\) matrix of the \(L\)
discrete phenotypes of interest. Because the gender of the fossilized
stickleback fish is unobservable, we further define
\(\boldsymbol{W} = (\boldsymbol{W}_{extant}^T,\boldsymbol{W}_{fossil}^T)^T\)
where \(\boldsymbol{W}_{extant}\) and \(\boldsymbol{W}_{fossil}\) are
the \(n_{extant} \times 1\) and \(n_{fossil} \times 1\) vectors of the
observed extant gender and missing fossil gender, respectively.

We impute the missing gender for the fossil data by sampling from the
posterior predictive distribution
\(P(\boldsymbol{W}_{fossil}|\boldsymbol{W}_{extant}, \boldsymbol{X}, \boldsymbol{Y})\)
using the multiple imputation by chained equations (MICE) algorithm
(Buuren and Groothuis-Oudshoorn (\protect\hyperlink{ref-MICE}{2011}))
with predictive mean matching. Traditionally, the choice for the number
of completed dats sets is a relatively small number such as \(M = 5\) or
\(M = 10\). However, Zhou and Reiter
(\protect\hyperlink{ref-ZhouReiter2009}{2010}) recommend a larger number
of imputed data sets if the data users intend on performing Bayesians
analysis after imputation, which in this case, we do. Therefore, the
imputation algorithm is run to obtain a total of \(M = 100\) completed
datasets. In addition to this, Zhou and Reiter
(\protect\hyperlink{ref-ZhouReiter2009}{2010}) suggests rather than
using Rubin's combining rules to combine across imputed data sets,
instead pool all of the draws from the posterior distributions across
all of the imputed data sets to estimate the posterior distributions of
parameters of interest. We proceed with our Bayesian analysis in this
manner.

\textcolor{red}{Akhil's stuff about validating the imputation model goes here.}

\hypertarget{completed-data-model}{%
\subsection{Completed Data Model}\label{completed-data-model}}

\textcolor{red}{We should note somewhere that we are only using the $W_{fossil}$ in the modeling part.  We drop the $W_{extant}$.  So just note that $W_{ti}$ is really $W_{fossil,ti}$.  Not sure how to say this, but we need to make it clear that we are only using the fossil data for the OU modeling.   }

\hypertarget{continuous-phenotypes}{%
\subsubsection{Continuous Phenotypes}\label{continuous-phenotypes}}

For a given imputed dataset, let \(W_{ti}\) be the imputed gender,
\(\boldsymbol{X}_{ti}\) be the \(K \times 1\) vector of continuous
phenotypes, and \(\boldsymbol{Y}_{ti}\) be the \(L \times 1\) vector of
discrete phenotypes for \(t = 1, \ldots, T\) and \(i = 1\ldots,n_{t}\)).

Define the last element of \(\boldsymbol{X}_{ti}\) (\(X_{K,ti}\)) as the
length of the stickleback. For \(k = 1,\cdots,K - 1\), we will assume
\begin{align}
{X}_{k,ti} & \overset{iid}{\sim}\left\{\begin{array}{llll} \mathcal{N}(\mu_{k,ft} & + \beta_k(X_{K,ti} - \mu_{K,ft}),&\sigma_k^2), & W_{ti} = \text{Female} \\ \mathcal{N}(\mu_{k,mt} & + \beta_k(X_{K,ti} - \mu_{K,mt}),&\sigma_k^2), & W_{ti} = \text{Male} \end{array}\right. \\
{X}_{K,ti} & \overset{iid}{\sim}\left\{\begin{array}{lll} \mathcal{N}(\mu_{K,ft},&\sigma_{K}^2), & W_{ti} = \text{Female} \\ \mathcal{N}(\mu_{K,mt},&\sigma_{K}^2), & W_{ti} = \text{Male} \end{array}\right.
\label{eq:X}
\end{align} where \(\mu_{k,ft}\) and \(\mu_{k,mt}\) represent the
time-\(t\) specific mean of phenotype \(k\) for female and male
stickleback fish, respectively, and \(\beta_k\) is an additional
parameter to account for the biological phenomenon that other continuous
phenotypes for a particular animal are positively correlated with their
length \textbf{CITATION NEEDED}. We further set \begin{align}
\mu_{k,gt} = \theta_{k,g} + u_{k,gt},
\label{eq:mu}
\end{align} for \(g \in \{f,m\}\) where \(\theta_{k,g}\) is the overall
mean of phenotype \(X_k\) for each gender, and \(u_{k,gt}\) measures the
difference between \(\mu_{k,gt}\) and \(\theta_{k,g}\). We then fit
\(\boldsymbol{u}_{k,g} = \{u_{k,g1},\cdots,u_{k,gT}\}\) to an
Ornstein-Uhlenback (OU) process (Uhlenbeck and Ornstein
(\protect\hyperlink{ref-OUProcess}{1930})) where we assume \(u_{k,gT}\)
have a marginal mean of 0. Because each time period is
\textasciitilde1000 years, we will discretize the OU process without
loss of generality. For \(t = 2,\ldots,T\), we assume \begin{align}
u_{k,gt} \overset{iid}{\sim}\mathcal{N}(\kappa_{k} u_{k,g(t-1)} , \tau_k^2).
\label{eq:u_ar1}
\end{align}

\(\kappa_{k,g}\) represents the correlation between \(\mu_{k,gt}\) and
\(\mu_{k,g(t+1)}\), and it also follows that
\(\text{cor}(u_{g,t},u_{g,t+h}) = \kappa^h\) for \(h \in \mathbb{N}\).
In addition, we assume \begin{align}
u_{k,g1} \overset{iid}{\sim}\mathcal{N}\left(0,\frac{\tau_k^2}{1-\kappa_{k}^2}\right).
\label{eq:u1}
\end{align} This model choice is to preserve stationarity;
i.e.~\(p(u_{k,gs}) = p(u_{k,gt})\) for \(s \neq t\).

\hypertarget{discrete-phenotypes}{%
\subsubsection{Discrete Phenotypes}\label{discrete-phenotypes}}

For the discrete phenotypes, we will assume \begin{align}
{Y}_{l,ti} & \sim \left\{\begin{array}{ll} Poisson(\lambda_{l,ft}), & W_{ti} = \text{Female} \\ Poisson(\lambda_{l,mt}), & W_{ti} = \text{Male} \end{array}\right.,
\label{eq:Y}
\end{align} for \(l = 1,\cdots,L\). Similar to the continuous
phenotypes, we set \begin{align}
\log(\lambda_{l,gt}) = \gamma_{l,g} + v_{l,gt}.
\label{eq:lambda}
\end{align} We use the log function to allow \(v_{l,gt}\) to take all
values on the real number line so we can properly fit an OU process to
these effects. We further assume \begin{align}
v_{l,gt} \overset{iid}{\sim}\mathcal{N}(\phi_{l} v_{l,g(t-1)} , \omega_l^2),
\label{eq:v_ar1}
\end{align} and \begin{align}
v_{l,g1} \overset{iid}{\sim}\mathcal{N}\left(0,\frac{\omega_l^2}{1 - \phi_{l}^2}\right)..
\label{eq:v1}
\end{align}

Because we are fitting a dataset with a stochastic structure on the
means of the phenotypes, we analyze the data via a Bayesian analysis.
Bayesian data analysis is also more naturally used when we have to
impute data (\textbf{CITATION}).

Priors: For \(k = 1,\cdots,K\) and \(l = 1,\cdots,L\),

\begin{align}
\sigma_k & \overset{iid}{\sim}\mathcal{N}(0,100)I_{\{\sigma > 0\}} \nonumber \\
\tau_k & \overset{iid}{\sim}\mathcal{N}(0,100)I_{\{\tau > 0\}} \nonumber \\
\kappa_k & \overset{iid}{\sim}\mathcal{N}(0,1)I_{\{-1 < \kappa_g < 1\}} \nonumber \\
\beta_k & \overset{iid}{\sim}\mathcal{N}(0,5) \nonumber \\
\theta_{k,g} & \overset{iid}{\sim}\mathcal{N}(0,10000) \nonumber \\
\omega_l & \overset{iid}{\sim}\mathcal{N}(0,100)I_{\{\tau > 0\}} \nonumber \\
\phi_l & \overset{iid}{\sim}\mathcal{N}(0,1)I_{\{-1 < \kappa_g < 1\}} \nonumber \\
\gamma_{l,g} & \overset{iid}{\sim}\mathcal{N}(0,10000)
\label{eq:priors}
\end{align}

All models were built using R Core Team
(\protect\hyperlink{ref-R2022language}{2022})

Cornuault (\protect\hyperlink{ref-Cornault2022}{2022}) Bayesian OU
model.

Bayesian Analysis after multiple imputation Zhou and Reiter
(\protect\hyperlink{ref-ZhouReiter2009}{2010}): They recommend using a
large number of imputations. 5 or 10 is too small. We are using M = 100.

\hypertarget{sec:results}{%
\section{Results}\label{sec:results}}

\hypertarget{sec:conclusions}{%
\section{Discsusson, Future work and
conclusions}\label{sec:conclusions}}

We predicted that release from predators would result in niche expansion
and increased sexual dimorphism, based on several studies of modern
stickleback. For example, in lakes where sculpin competitors are absent
and stickleback (Roesti et al.~2023) See Spoljaric and Reimchen 2008,
page 512 right column for references and discussion of differences
between benthic males and limnetic females. Male stickleback are benthic
and littoral (Wootton 1976)\ldots. Reimchen papers in general good for
this section.

\hypertarget{acknowledgements}{%
\section*{Acknowledgements}\label{acknowledgements}}
\addcontentsline{toc}{section}{Acknowledgements}

We thank O. Abughoush, S. Blaine, A. Chaudhary, M. Islam, F. Joaquin, C.
Lawson-Weinert, R. Sullivan, J. Tien, M.P. Travis, and W. Shim for help
with data collection. We thank D. Schluter and S. Blain for loaning
specimens and sharing data. We thank K. Swagel and C. McMahan of the
Field Museum for assistance with specimen x-rays. This research was
supported by NSF grants BSR-8111013, EAR-9870337, and DEB-0322818, the
Center for Field Research (Earthwatch), and the National Geographic
Society (2869-84) to MAB. It was also supported by NSF grants
DEB-1456462 and EAR-2145830 to YES. And NSF DMS-2015374 (GJM)

\hypertarget{supplementary-material}{%
\section*{Supplementary Material}\label{supplementary-material}}
\addcontentsline{toc}{section}{Supplementary Material}

All code for reproducing the analyses in this paper is publicly
available at \url{https://github.com/Akhil-Ghosh/SticklebackProject}

\hypertarget{references}{%
\section*{References}\label{references}}
\addcontentsline{toc}{section}{References}

\hypertarget{refs}{}
\begin{CSLReferences}{1}{0}
\leavevmode\vadjust pre{\hypertarget{ref-Aguirre2014}{}}%
Aguirre, Windsor E, Kendal Walker, and Shawn Gideon. 2014. {``Tinkering
with the Axial Skeleton: Vertebral Number Variation in Ecologically
Divergent Threespine Stickleback Populations.''} \emph{Biol. J. Linn.
Soc. Lond.} 113 (1): 204--19.

\leavevmode\vadjust pre{\hypertarget{ref-Bell2009}{}}%
Bell, M A. 2009. {``Implications of a Fossil Stickleback Assemblage for
Darwinian Gradualism.''} \emph{J. Fish Biol.} 75 (8): 1977--99.

\leavevmode\vadjust pre{\hypertarget{ref-Bell2006}{}}%
Bell, Michael A, Matthew P Travis, and D Max Blouw. 2006. {``Inferring
Natural Selection in a Fossil Threespine Stickleback.''}
\emph{Paleobiology} 32 (4): 562--77.

\leavevmode\vadjust pre{\hypertarget{ref-Blain2022}{}}%
Blain, S A. 2022. \emph{Evolutionary Outcomes of Interactions Among
Phenotypes in Post-Glacial Lakes}. University of British Columbia,
Canada.

\leavevmode\vadjust pre{\hypertarget{ref-Bolnick2011}{}}%
Bolnick, Daniel I. 2011. {``Sympatric Speciation in Threespine
Stickleback: Why Not?''} \emph{Int. J. Ecol.} 2011: 1--15.

\leavevmode\vadjust pre{\hypertarget{ref-Bolnick2003}{}}%
Bolnick, Daniel I, and Michael Doebeli. 2003. {``Sexual Dimorphism and
Adaptive Speciation: Two Sides of the Same Ecological Coin.''}
\emph{Evolution} 57 (11): 2433--49.

\leavevmode\vadjust pre{\hypertarget{ref-Bolnick2010}{}}%
Bolnick, Daniel I, Travis Ingram, William E Stutz, Lisa K Snowberg, On
Lee Lau, and Jeff S Paull. 2010. {``Ecological Release from
Interspecific Competition Leads to Decoupled Changes in Population and
Individual Niche Width.''} \emph{Proc. Biol. Sci.} 277 (1689): 1789--97.

\leavevmode\vadjust pre{\hypertarget{ref-Bolnick2008}{}}%
Bolnick, Daniel I, and On Lee Lau. 2008. {``Predictable Patterns of
Disruptive Selection in Stickleback in Postglacial Lakes.''} \emph{Am.
Nat.} 172 (1): 1--11.

\leavevmode\vadjust pre{\hypertarget{ref-Bowne1994}{}}%
Bowne, P S. 1994. {``Systematics and Morphology of the
Gasterosteiformes.''} In \emph{The Evolutionary Biology of the
Threespine Stickleback}, edited by M A Bell and S A Foster. Oxford, UK:
Oxford University Press.

\leavevmode\vadjust pre{\hypertarget{ref-Butler2007}{}}%
Butler, Marguerite A, Stanley A Sawyer, and Jonathan B Losos. 2007.
{``Sexual Dimorphism and Adaptive Radiation in Anolis Lizards.''}
\emph{Nature} 447 (7141): 202--5.

\leavevmode\vadjust pre{\hypertarget{ref-MICE}{}}%
Buuren, Stef van, and Karin Groothuis-Oudshoorn. 2011. {``Mice:
Multivariate Imputation by Chained Equations in r.''} \emph{Journal of
Statistical Software} 45 (3): 1--67.
\url{https://doi.org/10.18637/jss.v045.i03}.

\leavevmode\vadjust pre{\hypertarget{ref-Cooper2011}{}}%
Cooper, Idelle A, R Tucker Gilman, and Janette Wenrick Boughman. 2011.
{``Sexual Dimorphism and Speciation on Two Ecological Coins: Patterns
from Nature and Theoretical Predictions.''} \emph{Evolution} 65 (9):
2553--71.

\leavevmode\vadjust pre{\hypertarget{ref-Cornault2022}{}}%
Cornuault, Josselin. 2022. {``{Bayesian Analyses of Comparative Data
with the Ornstein--Uhlenbeck Model: Potential Pitfalls}.''}
\emph{Systematic Biology} 71 (6): 1524--40.
\url{https://doi.org/10.1093/sysbio/syac036}.

\leavevmode\vadjust pre{\hypertarget{ref-Hendry2009}{}}%
Hendry, A P, D I Bolnick, D Berner, and C L Peichel. 2009. {``Along the
Speciation Continuum in Sticklebacks.''} \emph{J. Fish Biol.} 75 (8):
2000--2036.

\leavevmode\vadjust pre{\hypertarget{ref-Herrmann2021}{}}%
Herrmann, Nicholas C, James T Stroud, and Jonathan B Losos. 2021. {``The
Evolution of 'Ecological Release' into the 21st Century.''} \emph{Trends
Ecol. Evol.} 36 (3): 206--15.

\leavevmode\vadjust pre{\hypertarget{ref-Hone2017}{}}%
Hone, David W E, and Jordan C Mallon. 2017. {``Protracted Growth Impedes
the Detection of Sexual Dimorphism in Non-Avian Dinosaurs.''}
\emph{Palaeontology} 60 (4): 535--45.

\leavevmode\vadjust pre{\hypertarget{ref-Hunt2008}{}}%
Hunt, Gene, Michael A Bell, and Matthew P Travis. 2008. {``Evolution
Toward a New Adaptive Optimum: Phenotypic Evolution in a Fossil
Stickleback Lineage.''} \emph{Evolution} 62 (3): 700--710.

\leavevmode\vadjust pre{\hypertarget{ref-little2002statistical}{}}%
Little, R. J. A., and D. B. Rubin. 2002. \emph{Statistical Analysis with
Missing Data}. Wiley Series in Probability and Mathematical Statistics.
Probability and Mathematical Statistics. Wiley.
\url{http://books.google.com/books?id=aYPwAAAAMAAJ}.

\leavevmode\vadjust pre{\hypertarget{ref-R2022language}{}}%
R Core Team. 2022. \emph{R: A Language and Environment for Statistical
Computing}. Vienna, Austria: R Foundation for Statistical Computing.
\url{https://www.R-project.org/}.

\leavevmode\vadjust pre{\hypertarget{ref-SaittaEtAl2020}{}}%
Saitta, Evan T, Maximilian T Stockdale, Nicholas R Longrich, Vincent
Bonhomme, Michael J Benton, Innes C Cuthill, and Peter J Makovicky.
2020. {``{An effect size statistical framework for investigating sexual
dimorphism in non-avian dinosaurs and other extinct taxa}.''}
\emph{Biological Journal of the Linnean Society} 131 (2): 231--73.
\url{https://doi.org/10.1093/biolinnean/blaa105}.

\leavevmode\vadjust pre{\hypertarget{ref-Schluter1992}{}}%
Schluter, D, and J D McPhail. 1992. {``Ecological Character Displacement
and Speciation in Sticklebacks.''} \emph{Am. Nat.} 140 (1): 85--108.

\leavevmode\vadjust pre{\hypertarget{ref-Shine1989}{}}%
Shine, R. 1989. {``Ecological Causes for the Evolution of Sexual
Dimorphism: A Review of the Evidence.''} \emph{Q. Rev. Biol.} 64 (4):
419--61.

\leavevmode\vadjust pre{\hypertarget{ref-Stuart2021}{}}%
Stuart, Yoel E, J William Sherwin, Ambika Kamath, and Thor Veen. 2021.
{``Male and Female Anolis Carolinensis Maintain Their Dimorphism Despite
the Presence of Novel Interspecific Competition.''} \emph{Evolution} 75
(11): 2708--16.

\leavevmode\vadjust pre{\hypertarget{ref-Stuart2020}{}}%
Stuart, Yoel E, Matthew P Travis, and Michael A Bell. 2020. {``Inferred
Genetic Architecture Underlying Evolution in a Fossil Stickleback
Lineage.''} \emph{Nat. Ecol. Evol.} 4 (11): 1549--57.

\leavevmode\vadjust pre{\hypertarget{ref-Stuart2017}{}}%
Stuart, Yoel E, Thor Veen, Jesse N Weber, Dieta Hanson, Mark Ravinet,
Brian K Lohman, Cole J Thompson, et al. 2017. {``Contrasting Effects of
Environment and Genetics Generate a Continuum of Parallel Evolution.''}
\emph{Nat. Ecol. Evol.} 1 (6): 158.

\leavevmode\vadjust pre{\hypertarget{ref-OUProcess}{}}%
Uhlenbeck, G. E., and L. S. Ornstein. 1930. {``On the Theory of the
Brownian Motion.''} \emph{Phys. Rev.} 36 (September): 823--41.
\url{https://doi.org/10.1103/PhysRev.36.823}.

\leavevmode\vadjust pre{\hypertarget{ref-Voje2022}{}}%
Voje, Kjetil L, Michael A Bell, and Yoel E Stuart. 2022. {``Evolution of
Static Allometry and Constraint on Evolutionary Allometry in a Fossil
Stickleback.''} \emph{J. Evol. Biol.} 35 (3): 423--38.

\leavevmode\vadjust pre{\hypertarget{ref-ZhouReiter2009}{}}%
Zhou, X., and J. Reiter. 2010. {``A Note on Bayesian Inference After
Multiple Imputation.''} \emph{The American Statistician} 64 (2):
159--63.

\end{CSLReferences}

\end{document}
